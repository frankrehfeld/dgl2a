\lecture{1}{18.04.2017}{Dr. Raphael Kruse}{Frank Rehfeld}
\section{Verallgemeinerte Ableitungen im Eindimensionalen}

\textbf{Motivation}\\
$u\colon\W\to\R$ gesucht mit
\begin{align*}
	-&u^{\prime\prime}\left(x\right) = f\left(x\right) \forall x\in\Omega\\
	 &u\left(a\right) = u\left(b\right) = 0 \forall x\in\partial\Omega
\end{align*}
für ein gegebenes $f$

\textbf{Ziel}\\
Lösungsbegriff abschwächen um unstetige $f$ umzusetzen. Idee hierfür ist die Multiplikation mit Testfunktionen und anschließende partielle Integration.
\begin{equation*}
	\int_{a}^{b} u^{\prime}\left(x\right)v^{\prime}\left(x\right) + \cancel{\left[u^{\prime}\left(x\right)v\left(x\right)\right]_{a}^{b}} = \int_{a}^{b} f\left(x\right)v\left(x\right)
\end{equation*}
Somit ist nicht mehr nötig, dass $u\in\C^{2}$. Dafür zusätzliche Forderung: $u^{\prime}v^{\prime}, fv\in\L^{1}$
