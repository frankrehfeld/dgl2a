\lecture{12}{24.05.2017}{Dr. Raphael Kruse}{Frank Rehfeld}
\begin{proof}[Fortsetzung]
	\underline{starke Monotonie}
	
	Betrachte
	\begin{equation*}
		\left<A\left[u\right]-A\left[v\right],u-v\right> = \int_{\W}\left(\Psi\left(|u^{\prime}|\right)u^{\prime}-\Psi\left(|v^{\prime}|\right)v^{\prime}\right)\left(u^{\prime}-v^{\prime}\right)\d x
	\end{equation*}
	Wir unterscheiden die Fälle
	\begin{enumerate}
		\item $x\in\W\colon u^{\prime}\left(x\right)\geq v^{\prime}\left(x\right)\geq 0$: Mit Eigenschaft 3. folgt
		\begin{equation*}
			\left(\Psi\left(|u^{\prime}\left(x\right)|\right)u^{\prime}\left(x\right) - \Psi\left(|v^{\prime}\left(x\right)|\right)v^{\prime}\left(x\right)\right)\left(u^{\prime}\left(x\right)-v^{\prime}\left(x\right)\right) \geq m\left(u^{\prime}\left(x\right)-v^{\prime}\left(x\right)\right)^{2}
		\end{equation*}
		\item $x\in\W\colon v^{\prime}\left(x\right)\geq u^{\prime}\left(x\right)\geq 0$ folgt analog mit vertauschten Rollen
		\item $x\in\W\colon v^{\prime}\left(x\right)\leq u^{\prime}\left(x\right)\leq 0$ und $x\in\W\colon u^{\prime}\left(x\right)\leq v^{\prime}\left(x\right)\leq 0$ folgen analog
		\item $x\in\W\colon v^{\prime}\left(x\right)\leq 0 \leq u^{\prime}\left(x\right)${}
		\begin{align*}
			&\left(\Psi\left(|u^{\prime}\left(x\right)|\right)u^{\prime}\left(x\right)-\Psi\left(|v^{\prime}\left(x\right)|\right)v^{\prime}\left(x\right)\right)\left(u^{\prime}\left(x\right)-v^{\prime}\left(x\right)\right)\\ 
			= &\underbrace{\Psi\left(-u^{\prime}\left(x\right)\right)}_{\geq m}\underbrace{u^{\prime}\left(x\right)}{\leq 0}\underbrace{\left(u^{\prime}\left(x\right)-v\left(x\right)\right)}_{\leq 0} - \underbrace{\Psi\left(-v^{\prime}\left(x\right)\right)}_{\geq m}\underbrace{v^{\prime}\left(x\right)}{\geq 0}\underbrace{\left(u^{\prime}\left(x\right)-v\left(x\right)\right)}_{\geq 0} \\
			\geq &m\left(u^{\prime}\left(x\right)-v^{\prime}\left(x\right)\right)^{2}
		\end{align*}
		\item $x\in\W\colon u^{\prime}\left(x\right)\leq 0 \leq v^{\prime}\left(x\right)$ analog zu 4.
	\end{enumerate}
	Somit ist $\Psi$ stark monoton.
\end{proof}

\textbf{Für nichtlineare Probleme ist kein allgemein gültiger Regularitätssatz bekannt.}

\begin{lemma}[Eindeutigkeit]
	$A\colon V\to V^{\prime}$ stark monoton $\implies A$ injektiv.
\end{lemma}
\begin{proof}
	Seien $v_{1},v_{2}\in V\colon A\left[v_{1}\right] = A\left[v_{2}\right]$. Dann gilt
	\begin{equation*}
		0 = \left<A\left[v_{1}\right]-A\left[v_{2}\right],v_{1}-v_{2}\right> \geq \mu\|v_{1}-v_{2}\|^{2}
	\end{equation*}
	Somit folgt $v_{1}=v_{2}$.
\end{proof}

\section{Galerkin-Verfahren und FEM}
\begin{definition}
	Sei $\left(V,\|\cdot\|_{V}\right)$ ein reeller Banachraum und $\left(V_{m}\right)_{m\in\N}$ eine Folge von endlichdimensionalen Teilräumen. $\left(V_{m}\right)$ heißt \textbf{Galerkin-Schema}, falls $\lim_{m\to\infty} \operatorname{d}\left(v,V_{m}\right) = 0 \forall v\in V$. Diese Eigenschaft wird \textbf{limitierte Vollständigkeit} genannt.
\end{definition}

\begin{definition}
	Sei $\left(\phi_{j}\right)_{j\in I}\subset V$, $I\subset \N$ mit
	\begin{itemize}
		\item je endlich viele Elemente von $\left(\phi_{j}\right)$ sind linear unabhängig
		\item $V_{m} := \operatorname{span}\left\{\phi_{1},\dots,\phi_{m}\right\}, m\in I$, ist ein Galerkin-Schema
	\end{itemize}
	Dann wird $\left(\phi_{j}\right)$ \textbf{Galerkin-Basis} von $V$ genannt.
\end{definition}

\begin{lemma}
	Sei $V$ ein reeller, separabler Banachraum. Dann besitzt $V$ eine Galerkin-Basis.
\end{lemma}
\begin{proof}
	\begin{enumerate}
		\item $\operatorname{dim} V<\infty$: Dann ist jede Basis von $V$ eine Galerkin-Basis.
		\item $\operatorname{dim} V=\infty$: Da $V$ separabel ist existiert eine dichte Teilmenge $\left(\psi_{j}\right)_{j\in\N}$.\\
			Setze nun
			\begin{align*}
				\phi_{1} :&= \psi_{1} & V_{1} &= \operatorname{span}\left\{\phi_{1}\right\}\\
				\phi_{m} :&= \psi_{j\left(m\right)} & j\left(m\right) &= \operatorname{min}\left\{j\in\N\colon \psi_{j}\notin V_{m-1}\right\}
			\end{align*}
			und es folgt
			\begin{equation*}
				\left(\psi_{j}\right)_{j=1}^{j\left(m\right)}\subset V_{m} = \operatorname{span}\left\{\phi_{1},\dots,\phi_{m}\right\}
			\end{equation*}
			und somit auch
			\begin{equation*}
				\left(\psi_{j}\right)_{j\in\N}\subset \bigcup_{m=1}^{\infty} V_{m} \implies \overbar{\bigcup_{m=1}^{\infty}V_{m}} = V
			\end{equation*}
			Also gilt: $\forall \e>0,v\in V\exists\tilde{v}\in\bigcup_{m=1}^{\infty}\colon\|v-\tilde{v}\|_{V} < \e$ und $\exists m\in\N\colon\tilde{v}\in V_{m}\subset V_{\tilde{m}}\forall\tilde{m} \geq m$.
			
			Zusammen erhält man
			\begin{equation*}
				\operatorname{d}\left(v,V_{\tilde{m}}\right) = \operatorname{inf}_{w\in V_{\tilde{m}}}\|v-w\|_{V} \leq \|v-\tilde{v}\|_{V} < \e \forall\tilde{m}\geq m
			\end{equation*}
	\end{enumerate}
\end{proof}

Betrachtet man das Problem
\begin{equation*}
	\left(V\right)\begin{cases}
		\text{Finde }u\in V\text{ mit}\\
		a\left(u,v\right)=\left<f,v\right> \forall v\in V
	\end{cases}
\end{equation*}
mit $a$ wie in Kapitel 4. oder 5. Dann können wir $\left(V\right)$ mit einer Galerkin Dimensionsreduzierung in das Ersatzproblem
\begin{equation*}
	\left(V_{m}\right)\begin{cases}
		\text{Finde }u_{m}\in V_{m}\text{ mit}\\
		a\left(u_{m},v_{m}\right) = \left<f,v_{m}\right> \forall v_{m}\in V_{m}
	\end{cases}
\end{equation*}
überführen. 
\begin{enumerate}
	\item \underline{Lösbarkeit}\\
		Da $V_{m}$ ein Teilraum von $V$ ist, ist $V_{m}$ abgeschlossen und somit ein Banachraum. Demzufolge sind Lax-Milgram oder Zarantonello anwendbar um die Lösbarkeit zu erhalten.
\end{enumerate}
