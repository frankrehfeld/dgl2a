\lecture{6}{03.05.2017}{Dr. Raphael Kruse}{Frank Rehfeld}
\begin{proof}[Fortsetzung]
	\begin{enumerate}
		\setcounter{enumi}{3}
		\item Stetigkeit im $\L^{2}$-Mittel $\implies \forall\eta>0\exists\e_{\eta}\in\left(0,\delta\right)\colon\forall\e\in\left(0,\e_{\eta}\right)\colon\|v_{1,\e_{\eta}}-v_{1}\|_{1,2} < \eta$ in $\H^{1}\left(a,b\right)$.{}
		\item Zusammen ergibt sich für $\delta,\e$ hinreichend klein:
			\begin{equation*}
				\|v_{1,\e}-u_{1}\| \leq \|v_{1,\e}-v_{1}\|_{1,2}+\|v_{1}-u_{1}\|_{1,2}<2\eta
			\end{equation*}
		\item  $u_{3}$ wird analog approximiert, $u_{2}$ kann direkt geglättet werden
		\item $v_{\e} := \left(v_{1,\e}+v_{2,\e}+v_{3,\e}\right)\einschraenkung_{\left[a,b\right]}\in\C^{\infty}\left[a,b\right]$ mit
			\begin{equation*}
				\|v_{\e}-u\|_{1,2} \leq \sum_{i=1}^{3}\|v_{i,\e}-u_{i}\|_{1,2} < 6\eta
			\end{equation*}
	\end{enumerate}
\end{proof}

\begin{lemma}[Produktregel]
	$u\in\H^{1}\left(\W\right),\Psi\in\C^{\infty}_{\text{c}}\left(\W\right) \implies u\Psi\in\H^{1}\left(\W\right)$ und $\left(u\Psi\right)^{\prime}=u^{\prime}\Psi+u\Psi^{\prime}$
\end{lemma}
\begin{proof}
	$\Psi\in\C^{\infty}_{\text{c}}\left(\W\right)\implies u\Psi\in\L^{2}\left(\W\right)$ und $u^{\prime}\Psi+u\Psi^{\prime}\in\L^{2}\left(\W\right)$. Für $\phi\in\C^{\infty}_{\text{c}}\left(\W\right)$ gilt
	\begin{align*}
		\int_{\W}u\Psi\phi^{\prime}\d x &= \int_{\W}u\left[\left(\Psi\phi\right)^{\prime}-\Psi^{\prime}\phi\right]\d x\\
			&= -\int_{\W}u^{\prime}\left(\Psi\phi\right)\d x - \int_{\W}u\Psi^{\prime}\phi\d x\\
			&= -\int_{\W}\left(u^{\prime}\Psi-u\Psi^{\prime}\right)\phi\d x
	\end{align*}
	Zusammen folgt, dass $u\Psi\in\H^{1}\left(\W\right)$.
\end{proof}

\begin{lemma}[Satz von Rellich]
	Es gilt
	\begin{equation*}
		\H^{1}\left(\W\right)\cembed\L^{2}\left(\W\right)
	\end{equation*}
\end{lemma}
\newpage
\begin{proof}
	Es gilt $\C\left(\overbar{\W}\right)\hookrightarrow\L^{2}\left(\W\right)$, da
	\begin{align*}
		\|v\|_{0,2}^{2} &= \int_{\W}|v\left(x\right)|^{2}\d x\\
			&\leq \|v\|_{\infty}^{2}\int_{\W}1\d x\\
			&=\|v\|_{\infty}^{2}\left(b-a\right)
	\end{align*}
	Somit folgt
	\begin{equation*}
		\H^{1}\left(\W\right)\cembed\C\left(\overbar{\W}\right)\hookrightarrow\L^{2}\left(\W\right)
	\end{equation*}
\end{proof}
\begin{itemize}
	\item $\C^{k}\left(\W\right), k\geq 1$ auch dicht in $\H^{1}\left(\W\right)${}
	\item $\C^{\infty}_{\text{c}}$ \textbf{nicht} dicht in $\H^{1}\left(\W\right)$
\end{itemize}

\begin{definition}
	\begin{equation*}
		H^{1}_{0}\left(\W\right) := \overbar{\C^{\infty}_{\text{c}}\left(\W\right)}
	\end{equation*}
	wobei der Abschluss bezüglich $\|\cdot\|_{1,2}$ gebildet wird.
	
	Es gilt $\H^{1}_{0}\left(\W\right)\subsetneq\H^{1}\left(\W\right)${}
	
	Sei $\left(\phi_{n}\right)\subset\C^{\infty}_{\text{c}}\left(\W\right)$ mit $\|\phi_{n}-u\|_{1,2}\to 0$ für ein $u\in\H^{1}\left(\W\right)$. Dann hat $u$ einen absolut stetigen Repräsentanten und es gilt $\|\phi_{n}-u\|_{\infty}\to 0$ und somit $u\left(a\right)=\operatorname{lim}\phi_{n}\left(a\right)=0$ und analog $u\left(b\right)=0$.{}
\end{definition}

\begin{lemma}[Charakterisierung von $\H^{1}_{0}\left(\W\right)$]
	\begin{equation*}
		\H^{1}_{0}\left(\W\right) = \left\{u\in\H^{1}\left(\W\right)\colon u\left(a\right)=u\left(b\right)=0\right\}
	\end{equation*}
\end{lemma}

\begin{lemma}[Poincare-Friedrichs-Ungleichung]
	Sei $u\in\H^{1}_{0}\left(\W\right)$. Dann gilt $\|u\|_{0,2}\leq\frac{b-a}{\sqrt{2}}\|u^{\prime}\|_{0,2}$.
\end{lemma}
\textbf{Die Aussage gilt nicht für $H^{1}\left(\W\right)$!}
\begin{proof}
	\begin{align*}
		u\left(x\right) &= \underbrace{u\left(a\right)}_{= 0} + \int_{a}^{x}u^{\prime}\left(\xi\right)\d\xi \implies |u\left(x\right)|\leq\int_{a}^{x}|u\left(\xi\right)|\d\xi\\
		\implies \|u\|_{0,2}^{2} &\leq \int_{\W}\left(\int_{a}^{x}1|u^{\prime}\left(\xi\right)|\d\xi\right)^{2}\d x\\
			&\leq \int_{\W}\left(\int_{a}^{x}1\d\xi \int_{a}^{x}|u^{\prime}\left(\xi\right)|^{2}\d\xi\right)\d x\\
			&\leq \int_{\W}\left(x-a\right)\d x \int_{a}^{b}|u^{\prime}\left(\xi\right)|^{2}\d\xi\\
			&= \frac{1}{2}\left(b-a\right)^{2}\|u^{\prime}\|_{0,2}^{2}
	\end{align*}
\end{proof}
\begin{itemize}
	\item Die Konstante kann auf $\frac{b-a}{\pi}$ verbessert werden.
	\item $\|\cdot\|_{1,2}$ und $|\cdot|_{1,2} := \|u^{\prime}\|_{0,2}$ sind äquivalent auf $\H^{1}_{0}\left(\W\right)$.{}
	\item Auf $\H^{1}\left(\W\right)$ ist $|\cdot|_{1,2}$ nur eine Halbnorm, da sie Konstanten übersieht.
\end{itemize}

\begin{lemma}[Satz]
	$\left(\H^{1}_{0}\left(\W\right),|\cdot|_{1,2},\left[u,v\right]_{1,2}:=\int_{\W}u^{\prime}\left(\xi\right)v^{\prime}\left(\xi\right)\d\xi=\left(u^{\prime},v^{\prime}\right)_{0,2}\right)$ ist ein separabler Hilbertraum.
	
	Es gilt $\H^{1}_{0}\left(\W\right)\cembed\L^{2}\left(\W\right)$ und $\H^{1}_{0}\left(\W\right)\cembed\C\left(\overbar{\W}\right)$.
\end{lemma}

\begin{lemma}[Satz]
	$\H^{1}_{0}\left(\W\right)$ liegt dicht in $\L^{2}\left(\W\right)$. Dies folgt aus $\C^{\infty}_{\text{c}}\left(\W\right)$ dicht in $\L^{2}\left(\W\right)$.{}
	\begin{equation*}
		\C^{\infty}_{\text{c}}\left(\W\right) \subset \H^{1}_{0}\left(\W\right) \subset \L^{2}\left(\W\right){}
	\end{equation*}
\end{lemma}

\begin{definition}
	\begin{equation*}
		\H^{-1}\left(\W\right) := \left(\H^{1}_{0}\left(\W\right)\right)^{\prime}
	\end{equation*}
	\begin{itemize}
		\item $\H^{-1}\left(\W\right) \simeq \H^{1}_{0}\left(\W\right)$ mit dem Riesz'schen Darstellungssatz
		\item Gelfand-Tripel: $\H^{1}_{0}\left(\W\right)\dembed\L^{2}\left(\W\right)\simeq\left(\L^{2}\left(\W\right)\right)^{\prime}\dembed\H^{-1}\left(\W\right)$
	\end{itemize}
\end{definition}

\begin{lemma}[Satz]
	\begin{equation*}
		\|f\|_{-1,2} := \operatorname{sup}_{v\neq 0}\frac{|\left<f,v\right>|}{|v|_{1,2}}
	\end{equation*}
	definiert eine Norm auf $H^{-1}\left(\W\right)$. $\left(\H^{-1}\left(\W\right),\|\cdot\|_{-1,2}\right)$ bildet einen reflexiven Banachraum. 
	
	\begin{equation*}
		L^{2}\left(\W\right)\cembed\H^{-1}\left(\W\right){}
	\end{equation*}
	oder 
	\begin{equation*}
		\forall f\in\H^{-1}\left(\W\right)\exists u_{f}\in\L^{2}\left(\W\right)\colon\left<f,v\right>=\int_{\W}u_{f}v^{\prime}\d x \forall v\in\H^{1}_{0}\left(\W\right){}
	\end{equation*}
\end{lemma}
